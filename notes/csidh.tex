\documentclass[10pt]{article}
\usepackage{amsmath,amsthm,amssymb,graphicx,titling,mathrsfs,tikz,tikz-cd}
\usepackage{mathtools}
\usepackage{mleftright}
\usepackage{tikz-cd}
\usepackage{todonotes}
\usepackage{color}
\usepackage{stmaryrd}
\usepackage{hyperref}
\usepackage{pgfplots}
\usepackage{braket}

\usepackage{listings}

\usepackage{xcolor}
\hypersetup{
    colorlinks,
    linkcolor={blue!80!black},
    citecolor={blue!80!black},
    urlcolor={blue!80!black}
}

\lstset{ %
  backgroundcolor=\color{white},   % choose the background color; you must add \usepackage{color} or \usepackage{xcolor}; should come as last argument
  basicstyle=\footnotesize\ttfamily,        % the size of the fonts that are used for the code
  breakatwhitespace=false,         % sets if automatic breaks should only happen at whitespace
  breaklines=true,                 % sets automatic line breaking
  captionpos=b,                    % sets the caption-position to bottom
  commentstyle=\color{mygreen},    % comment style
  deletekeywords={...},            % if you want to delete keywords from the given language
  escapeinside={\%*}{*)},          % if you want to add LaTeX within your code
  extendedchars=true,              % lets you use non-ASCII characters; for 8-bits encodings only, does not work with UTF-8
  frame=single,                    % adds a frame around the code
  keepspaces=true,                 % keeps spaces in text, useful for keeping indentation of code (possibly needs columns=flexible)
  keywordstyle=\color{blue},       % keyword style
  morekeywords={sage},            % if you want to add more keywords to the set
  rulecolor=\color{black},         % if not set, the frame-color may be changed on line-breaks within not-black text (e.g. comments (green here))
  showspaces=false,                % show spaces everywhere adding particular underscores; it overrides 'showstringspaces'
  showstringspaces=false,          % underline spaces within strings only
  showtabs=false,                  % show tabs within strings adding particular underscores
  stepnumber=2,                    % the step between two line-numbers. If it's 1, each line will be numbered
  tabsize=2,                     % sets default tabsize to 2 spaces
  title=\lstname                   % show the filename of files included with \lstinputlisting; also try caption instead of title
}

\usepackage[ruled,vlined]{algorithm2e}

\theoremstyle{plain}
\newtheorem{theorem}{Theorem}[subsection]
\newtheorem{lemma}[theorem]{Lemma}
\newtheorem{corollary}[theorem]{Corollary}
\newtheorem{proposition}[theorem]{Proposition}
\newtheorem{remark}[theorem]{Remark}
\newtheorem{definition}[theorem]{Definition}
\theoremstyle{definition}
\newtheorem{example}[theorem]{Example}
\newtheorem{exercise}[theorem]{Exercise}
\newtheorem{problem}[theorem]{Problem}

\newcommand{\iso}{\cong}
\newcommand{\op}{\operatorname}
\newcommand{\normlin}{\trianglelefteq}
\newcommand{\ideal}{\trianglelefteq}
\newcommand{\N}{\mathbb{N}}
\newcommand{\Z}{\mathbb{Z}}
\newcommand{\Q}{\mathbb{Q}}
\newcommand{\R}{\mathbb{R}}
\newcommand{\C}{\mathbb{C}}
\newcommand{\A}{\mathbb{A}}
\newcommand{\F}{\mathbb{F}}
\newcommand{\HH}{\mathbb{H}}
\newcommand{\p}{\mathfrak{p}}

\newcommand{\cat}[1]{{\normalfont\textbf{#1}}}
\newcommand{\Ring}{\cat{Ring}}
\newcommand{\Top}{\cat{Top}}
\newcommand{\Vect}{\cat{Vect}}

\newcommand{\Hom}{\op{Hom}}
\newcommand{\Obj}{\op{Obj}}
\newcommand{\End}{\op{End}}
\newcommand{\Aut}{\op{Aut}}
\newcommand{\nrd}{\op{nrd}}
\newcommand{\trd}{\op{trd}}
\newcommand{\disc}{\op{disc}}
\newcommand{\discrd}{\op{discrd}}
\newcommand{\chr}{\op{char}}
\newcommand{\Frac}{\op{Frac}}
\newcommand{\Pl}{\op{Pl}}
\newcommand{\Ram}{\op{Ram}}
\newcommand{\Cls}{\op{Cls}}
\newcommand{\Gen}{\op{Gen}}
\newcommand{\Typ}{\op{Typ}}
\newcommand{\rk}{\op{rk}}

\newcommand{\defeq}{\vcentcolon=}

\setlength{\droptitle}{-7em}
\title{CSIDH Paper}
\author{Joel Laity}

\begin{document}
\maketitle
\tableofcontents

% \begin{definition}[]
%     Let \( p \in \Z \) be prime.
%     Let \( K / \Q \) be a quadratic field.
%     Then
%     \[
%         p\mathcal{O}_K =
%         \begin{cases}
%             \mathfrak{p}^2 \text{ or }    \\
%             \mathfrak{p} \text{ or }      \\
%             \mathfrak{p}_1\mathfrak{p}_2.
%         \end{cases}
%     \]
%     We say that \( p \) is ramified, inert or split in the respective cases.
% \end{definition}

\section*{Hidden shift algorithm for cyclic groups}

We will look at Section 4.2 in the paper.

We will show how to convert and algorithm for the hidden shift problem in \( \Z / 2^n \Z \) to an algorithm for the hidden shift problem in \( \Z / N\Z \) where \( N \) is any natural number.

We start with two functions \( f, g: \Z / N\Z \to ?? \) such that
\[
    f(x) = g(x+s)
\]
for some \( s \in \Z / N\Z \).
We then choose \( n \in \N \) minimal such that \( 2^n > cN \) where \( c \) is constant to be chosen later which will affect the probability of success. The paper describes how you can set up the quantum state
\[
    \frac{1}{\sqrt{2^{n+1}}} \sum_{x=0}^{2^n-1} \left( \ket{0}\ket{x}\ket{f(x \bmod{N})} + \ket{1}\ket{x}\ket{g(x \bmod{N})} \right).
\]
{
\color{red} In the paper the sum is up to \( 2^n \) and the normalization is \( \frac{1}{2^{(n+1) / 2}} \) which has to be wrong because the sum of the squares of the amplitudes is not 1.
Also in the paper the second term is not in the scope of the sum which is then contradicted by later equations.
I believe what I wrote is what they meant.
}
We then make a measurement of the output register, let's say the result of the measurement is \( d \). The input registers will be a superposition of states \( \ket{x} \) where \( x \) runs over all values such that \( f(x \bmod{N}) + g(x + \bmod{N}) = d\). If we assume that \( f+g : \Z / N\Z \to ?? \) is injective then we will get a state that looks like this
\[
    \frac{1}{\sqrt{X}}
    \left(
    \underbrace{\ket{0}\ket{x_1} + \ket{1}\ket{x_2}}_{\text{noise}} +
    \underbrace{\sum_{j=0}^{\lfloor 2^n / N \rfloor }
        \left(
        \ket{0}\ket{x + jN}+ \ket{1}\ket{x + jN + s} \right)}_{\text{what we want}}
    \right).
\]
{
\color{red} The bound on the sum should be \( 2^n - 1 \).
}
Where \( X \) is the number of terms in the parentheses.
We note that \( c \leq 2^n / N \leq 2c \) so \( 2+c \leq X \leq 2c \).
Confusingly, in this equation \( x \) is a particular solution to \( f(x) + g(x) = d \), which for convenience is assumed to satisfy \( 0 \leq x < N \), while in the previous displayed equation it was in the scope of the sum.
The values \( x_1 \) and \( x_2 \) are possible leftover values which satisy \( f(x_1) + g(x_2) = d \) but \( x_1 \) is in the interval \( cN \leq x_1 < 2^n  \) and \( x_2 = (x_1 + s) \bmod{2^n} \) (or \( x_2 \) is in the interval \( cN \leq x_2 < 2^n  \) and \( x_1 = x_2 - s \)).
It is possible there is no ``leftover'' solution \( x_1, x_2 \) in which case these terms won't appear.

\begin{center}
    (picture goes here)
\end{center}

We now apply a quantum fourier transform over the second register.
We first recall how the quantum fourier transform works.
It maps a quantum state
\[
    \sum_{x=0}^{2^n-1} a(x)\ket{x} \mapsto \sum_{x=0}^{2^n-1} A(x)\ket{x}
\]
where \( a: \Z / 2^n \Z \to \C \) is a function of norm 1 and \( A : \Z / 2^n \Z \to \C\) is the DFT of \( a \).
First we rearrange the above equation
\[
    \frac{1}{\sqrt{X}} \left(\ket{0}\left(\ket{x_1} + \sum_{j=0}^{\lfloor 2^n / N \rfloor }\ket{x + jN} \right) + \ket{1} \left(\ket{x_2} + \sum_{j=0}^{\lfloor 2^n / N \rfloor }\ket{x + jN + s}\right) \right) .
\]
We apply the QFT over the second register we get
\[
    \frac{1}{\sqrt{2^n}}\frac{1}{\sqrt{X}}\sum_{y = 0}^{2^n - 1} \left(  \left(\chi\left(\frac{x_1y}{ 2^n}\right) + \sum_{j = 0}^{\lfloor 2^n / N \rfloor } \chi\left(\frac{y(x+jN)}{2^n}\right) \right) \ket{0}  +  \left( \chi\left(\frac{x_2y}{ 2^n}\right) + \sum_{j = 0}^{\lfloor 2^n / N \rfloor } \chi\left(\frac{y(x+jN + s)}{2^n}\right)\right) \ket{1} \right)\ket{y} ,
\]
where \( \chi(x) = \exp(2\pi ix) \), \( i = \sqrt{-1} \).
Take a measurement on the second register and we get
\[
    \frac{1}{\sqrt{2^n}}\frac{1}{\sqrt{X}}\left(\chi\left(\frac{x_1y}{ 2^n}\right) + \sum_{j = 0}^{\lfloor 2^n / N \rfloor } \chi\left(\frac{y(x+jN)}{2^n}\right) \right) \ket{0}  +  \frac{1}{2^n}\left( \chi\left(\frac{x_2y}{ 2^n}\right) + \sum_{j = 0}^{\lfloor 2^n / N \rfloor } \chi\left(\frac{y(x+jN + s)}{2^n}\right)\right) \ket{1} ,
\]
for some fixed \( y \).
Provided \( 2^n / N \) is large enough the probability that we choose a particular \( y \) should depend only on the character sum, rather than the noise.
In fact we will later show that for half of the \( y \in \Z / 2^n \Z \) the character sum is larger than \( \frac{2}{\pi} \)
We regroup into ``noise'' and ``what we want'' again
\[
    \frac{1}{\sqrt{2^n}}\frac{1}{\sqrt{X}}
    \left(
    \chi\left(\frac{x_1y}{ 2^n}\right)\ket{0}
    + \chi\left(\frac{x_2y}{ 2^n}\right) \ket{1}
    + \sum_{j = 0}^{\lfloor 2^n / N \rfloor }
    \left(
        \chi\left(\frac{y(x+jN)}{2^n}\right) \ket{0}
        + \chi\left(\frac{y(x+jN + s)}{2^n}\right)\ket{1}
        \right)
    \right).
\]
We factor out a common term and pull it outside the sum
\[
    \frac{1}{\sqrt{2^n}}\frac{1}{\sqrt{X}} \left(\chi\left(\frac{x_1y}{ 2^n}\right)\ket{0} + \chi\left(\frac{x_2y}{ 2^n}\right) \ket{1} + \chi\left(\frac{xy}{2^n}\right)\sum_{j = 0}^{\lfloor 2^n / N \rfloor } \left(\chi\left(\frac{yjN}{2^n}\right) \ket{0}  +    \chi\left(\frac{y(jN + s)}{2^n}\right)  \ket{1}\right)\right).
\]
We factor out a common term inside the sum
\[
    \frac{1}{\sqrt{2^n}}\frac{1}{\sqrt{X}} \left(\chi\left(\frac{x_1y}{ 2^n}\right)\ket{0} + \chi\left(\frac{x_2y}{ 2^n}\right) \ket{1} + \chi\left(\frac{xy}{2^n}\right)
    \sum_{j = 0}^{\lfloor 2^n / N \rfloor }
    \left(
        \chi\left(\frac{yjN}{2^n}\right) \left( \ket{0}  +    \chi\left(\frac{sy}{2^n}\right)  \ket{1} \right) \right)\right)
\]
and rearrange parentheses
\[
    \frac{1}{\sqrt{2^n}}\frac{1}{\sqrt{X}} \left(\chi\left(\frac{x_1y}{ 2^n}\right)\ket{0}
    + \chi\left(\frac{x_2y}{ 2^n}\right) \ket{1}
    + \chi\left(\frac{xy}{2^n}\right)
    \left(
        \sum_{j = 0}^{\lfloor 2^n / N \rfloor }
        \chi\left(\frac{yjN}{2^n}\right)
        \right)
    \left( \ket{0}  +    \chi\left(\frac{sy}{2^n}\right)  \ket{1} \right)\right)
\]
We now wish to bound from below the magnitude of the \( \left( \ket{0}  +    \chi\left(\frac{sy}{2^n}\right)  \ket{1} \right) \) term.
{
\color{red} (I only vaguely understand why we need to do this.)
}
We have
\begin{align*}
    \left | \chi\left(\frac{xy}{2^n}\right)\left(\sum_{j = 0}^{\lfloor 2^n / N \rfloor }\chi\left(\frac{yjN}{2^n}\right)\right) \right|
      & =  \left | \sum_{j = 0}^{\lfloor 2^n / N \rfloor }\chi\left(\frac{yjN}{2^n}\right) \right|     \\
      & = \left | \sum_{j = 0}^{\lfloor 2^n / N \rfloor }\chi\left(\frac{yN}{2^n}\right)^j \right|     \\
    \intertext{using the formula for a geometric series we get}
      & = \left | \frac{1-\chi(yN / 2^n )^{\lceil 2^n / N \rceil}}{1 - \chi(yN/2^n)} \right|           \\
      & = \left | \frac{1-\chi(\lceil 2^n / N \rceil yN / 2^n )}{1 - \chi(yN/2^n)} \right|             \\
    \intertext{using the identity \( |1-\exp(i\theta)| = 2\sin(\theta / 2) \) we get }
      & = \left | \frac{\sin(\pi \lceil 2^n / N \rceil yN / 2^{n} )}{\sin( \pi yN/2^{n})} \right|      \\
    \intertext{now \( |\sin(x)| > \frac{2}{\pi}|x| \) for \( -\pi / 2 \leq x \leq \pi / 2 \) and \( |\sin(x)| \leq |x| \) for all \( x \in \R \)} \\
      & \geq \left | \frac{\frac{2}{\pi} \pi \lceil 2^n / N \rceil yN / 2^{n} }{ \pi yN/2^{n}} \right| \\
      & = \frac{2}{\pi} \lceil 2^n / N \rceil                                                          \\
      & \geq \frac{2}{\pi} c,
\end{align*}
where \( c \) was the constant we chose before so that \( 2^n > cN \).
This inequality holds provided 
\[
-\pi / 2 \leq  \pi \lceil 2^n / N \rceil yN / 2^{n} \bmod{2\pi}   \leq \pi / 2 \Leftrightarrow \left| \lceil 2^n / N \rceil yN\right| \bmod{2^n} \leq 2^n/2
\]
, which is be true for half of the values of \( y \).
Recall that we chose \( n \) so that \( n \) was minimal with \( 2^n > cN \) so the final lower bound is
\[
    \frac{2}{\pi}c.
\]
We now multiply by the normalization factors we dropped and get
\[
    \frac{1}{\sqrt{2^n}}\frac{1}{\sqrt{X}}\frac{2}{\pi}c 
    \geq \frac{1}{\sqrt{2^n}}\frac{1}{\sqrt{2c}}\frac{2}{\pi}c 
    =\frac{1}{\pi\sqrt{2^{n-1}}}\sqrt{c}.
\]
Which doesn't seem right because half the Fourier coefficents are bounded from below by this expression which means that the original function has norm
\[
	\frac{2^n}{2}\left(\frac{1}{\pi\sqrt{2^{n-1}}}\sqrt{c} \right)^2
	=\frac{1}{\pi}c
\]
so for sufficiently large values of \( c \) the function will have norm at least greater than \( 1 \).
But the normalization constant \( X \) was chosen so the function has norm 1.


\end{document}




































